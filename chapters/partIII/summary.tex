\part{Conclusions}\label{part:conclusions}

\chapter{Summary and Outlook}


Our sixth sense for balance plays a vital role in everyday life. For instance, the vestibulo-ocular reflex stabilizes our gaze and allows us to read messages on our phones while we are walking. Bilateral loss of vestibular function disrupts more than gaze stabilization and significantly reduces quality of life. Currently a vestibular implant is the most promising rehabilitation option for affected patients.

Within the European-US project CLONS, BVL patients were instrumented with modified cochlear implants that provided independent electrode sites for vestibular stimulation. In Part\,\ref{part:pamprm}, we examined four patients acutely with pulse rate and pulse amplitude modulation and we strongly recommend PAM as stimulation paradigm for initial implant activation. In Part\,\ref{part:vecap}, we pursued a more risky and bold goal of CLONS and made first steps towards a closed-loop VI with VECAP as feedback signal. Results of VECAP-VOR correlation were encouraging, but will require future effort. 

\section*{Scientific Contributions}
Regarding \textbf{Aim 1} of identifying efficient electrical stimulation paradigms for VI activation, we programmed a real-time research platform that can interface with any MED-EL cochlear implant with latencies less than \SI{10}{\milli\second}. Standard clinical tools, such as the company's Research Interface Box or Diagnostic Interface Box, have too large a latency and cannot interface with an external sensor such as a rotation sensor. The research platform could be also used for other scenarios where implanted electrodes in other parts of the body are activated by a cochlear implant.

The study design for Chapter\,\ref{chap:pamprm} injected identical charge with PAM and PRM and our experimental results and model simulations revealed that PAM evoked stronger responses than PRM. It is a clinically relevant finding that PAM is more efficient than PRM for VI activation. Surprisingly, our findings cast strong doubts on the common notion that same charge equates to same number of evoked action potentials which in turn evoke the same eye movement response. Specifically, the model revealed different ensemble firing rates for charge-equivalent PAM and PRM (higher for PAM). And even assuming identical ensemble firing rates for PAM and PRM, PAM taps afferents with stronger synaptic weights than PRM leading to stronger eye movement responses.

Regarding \textbf{Aim 2} of characterizing VECAPs in animal models, we reported VECAPs in correlation with VOR responses for different stimulation scenarios. Techniques were described to record and reduce artifact with a multi-site, double-sided electrode array. The techniques could be useful if the electrode array were to be deployed in other cases. 

Second, we introduced the concept of a closed-loop VI with VECAP as feedback signal. A provisional US patent application for a closed-loop VI had been filed, but was not further pursued because of electrode issues and because results (from one guinea pig at the time) were not sufficiently compelling. However, by pursuing the audacious goal of a closed-loop VI, we found other opportunities for clinical applications of VECAP such as threshold estimation.

\section*{Outlook}
We are now on the cusp of a major step towards commercial VIs. In two to three years, the first patients could use a chronically active VI \emph{outside} a controlled laboratory environment. Some obstacles remain. Besides regulatory approval, a motion sensor has to be fixed to the subject's head and the processing unit of the cochlear implant has to be adapted to accommodate vestibular stimulation. These chronic tests will reveal how the subject and the subject's central pathways will adapt to the stimulation and could lead to the first generation of commercial VIs.

We promote VECAP as instrument to improve the use of VIs and future VI performance in the medium and long term (10 years). It has already been suggested as tool for electrode placement during surgery. A medium term step would be to include VECAP for implant fitting post surgery and also during regular follow-ups. This would be more efficient than the currently laborious protocol. In fact, this year we have started a collaboration with an industrial partner (MED-EL) to record VECAPs in human patients. The results will determine whether VECAPs can be recorded with the current vestibular implants. The outcome will also outline possible modifications for better VECAP recording and discuss steps to put VECAPs into clinical use. In the long term, implementing VECAP as feedback signal for a closed-loop VI could boost implant performance by providing a consistent input to the VOR and other pathways. 

The vestibular implant will provide BVL patients with a treatment option to restore vestibular function and quality of life. With the words of Immanuel Kant, their sense of balance should not err anymore.


% TODO a figure to unify the two chapters here or later in part iii, better part iii
%Parts\,\ref{part:pamprm} and \ref{part:pamprm} of the thesis had in common that both first required  technological developments (real-time research platform, electrode array, artifact removal) to address the intended research question. 