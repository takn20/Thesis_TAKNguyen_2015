\chapter{Background}\label{sec:background}
This section provides an overview of VI research in animal models as well as human patients. A handful of research groups have excelled demonstrated expertise, most prominent merfeld and lewis, della santina, rubinstein and philipps, guyot, perez-fornos, kingma. Comprehensive reviews have been published recently (Merfeld and Lewis, van de Berg, Fridman and Della Santina). A table in appendix gives overview of research, similar to table in van de Berg. More general vestibular neurophysiology are excellently detailed in The Vestibular System and Clinical Neurophysiology of the Vestibular System. 

Introduce general concept here. Motion sensor senses head movement, processor/ controller computes stimulation strategy and parameters. Sends parameters to nerve stimulator to vestibular nerve. This thesis will focus on devices that aim to restore SCC function. This has been the main avenue of research, because each SCC encodes a distinct axis. In contrast, utricle and sacule encode different directions and the remainder of this thesis will 
TODO figure

General device description for most protoypes in animal models
In most VI prototypes, only electrodes implanted (transcutaneously) 
A gyroscopic sensor attached to the head to measure head rotation. Signal filtered and digitized and transferred to a processing unit (typically a microcontroller) that determines stimulation parameters and sends this to a current source. The latter then applies the stimulation pulse through implanted electrode sites.

Motion sensor
First devices employed single axis MEMS gyroscopes or coriolis vibratory gyroscope (Gong and Merfeld). What is MEMS, basic concept how it works. Combining three of them for Della Santina's three single axis one but perpendicularly aligned to each other. With advent in MEMS technology sensors smaller, higher range and sensitivity (Gong 2000 1.128 mV/deg/s). Major challenge is power consumption. Due to their working principle (actively oscillating elements) draw current even when inactive. Now with standby state. Merfeld model Gyrostar(R) ENC-05Eone axis drew 2.5-5mA, the ITG-3200, standby current 5uA, 6.5mA operating consumption. To compare triple axis accelerometer uses 25-130uA. Developments range, power consumption, size; even smaller, or biomimetic sensors. And where to place Med-El patent has in-ear, outside on skull or implanted.

Electrode design and placement


Stimulation paradigm


Gong and Merfeld 2000 proved feasibility of single canal VI. A GP was instrumented with a platinum wire in which? canal. Acute electrical stimulation with these characteristics  evoked eye movement responses through the VOR. Magnitude of responses were large. Expansion to study VOR adaptation to baseline (Saginaw), squirrel and rhesus monkeys, chronic stimulation (Merfeld 2005, 2007). Also tested a bilateral implant (Gong 2008). What is their expertise? Diverse animal models, chronic and VOR assessment as well as some balance function (Thompson).

Laboratory of Vestibular Neurophysiology at JHU first to demonstrate a multichannel/multicanal VI or MVP. Animal model chinchilla, rendered BVL through intratympanic injection of ototoxic aminoglycoside gentamicin Detailed description of their first generation device in Dai, second generation here MVP2 Chiang 2011. Misalignment improvements (Fridman 2010), effects of pulse parameters. modulation paradigms (davidovics, another paper as well). continuous stimulation in chinchillas and recently in monkeys. Valentin et al papers on modified CIs for vestibular stimulation. Strength is 3D stimulation paradigm.

Group in Washington. A third group at University of Washington under the aegis of Rubinstein and Philipps have instrumented rhesus monkeys and also human patients with vestibular implant prototypes. Rubinstein 2010, 2011 (prosthetic implantation of semicircular canals). VECAP, Nie et al 2011, for intraoperative monitoring for placement, because VECAP not affected by anesthesia such as VOR Philipps 2011, Golub 2013 (human). longitudinal study and three electrodes per canal in patients.