\chapter{Background}\label{chap:background}
We begin this chapter with a review of basic vestibular neurophysiology. More general and comprehensive vestibular neurophysiology is covered at length in The Vestibular System: A Sixth Sense (Goldberg et al., 2012) or Clinical Neurophysiology of the Vestibular System (Baloh and Kerber, 2011). 

We then introduce the general VI design and describe its components such as motion sensor, electrodes and stimulation paradigms. For detailed reviews on the VI topic, see van de Berg (2011), Merfeld and Lewis (2012), or Fridman and Della Santina (2012). Readers familiar with semicircular canals, the vestibulo-ocular reflex and the VI concept may refresh on individual topics or view this chapter as reference.

\section{Structure and Physiology of the Vestibular System}\label{sec:background:physio}
\begin{figure}[btp]
\centering
\includegraphics[width=.8\textwidth]{chapters/background/figures/Fig_AnatomyPhysiology_v02-01.eps} 
\caption[Anatomy and physiology of the vestibular labyrinth]{Anatomy and physiology of the vestibular labyrinth. (\textbf{A}) Anatomy of the ear with a close-up of the inner ear with its vestibular and auditory organs. (\textbf{B}) Each semicircular canal has an ampulla with the shown elements. In case of a rotation, the endolymph inside the canal deflects the cupula and triggers a change in firing rates of nerve fibers. (\textbf{C}) An example of how one afferent changes firing rate to head velocity. Note that at zero velocity, the afferent still has a resting discharge rate of approx. 90\,spk/s. (\textbf{D}) The semicircular canals in both ears form pairs in three distinct axes: horizontal, LARP and RALP. (\textbf{E}) Regular and irregular afferents have different gains. Irregular afferents are more suited to encode high frequency movement. (\textbf{F}) The pathway of the horizontal VOR. A head turn to the right evokes excitation in the right horizontal SCC (firing rates increase) and inhibition in the left horizontal SCC. Through a three nuclei (or synapses) eyes respond with movement to the left to counter the head motion and stabilize gaze.
}
\label{fig:anatomy}
\end{figure}

Our peripheral vestibular system in the inner ear is a marvelous inertial sensor with a large dynamic range and high sensitivity. Figure\,\ref{fig:anatomy}A shows the division into outer, middle and inner ear. The latter consists of the membranous labyrinth that is contained within the bony labyrinth. The membranous labyrinth is filled with endolymph, an extracellular fluid with a distinctive, potassium-rich composition. It contributes to mechanotransduction, i.\,e. the conversion of head movement into nerve signals.

The membranous labyrinth includes the cochlea for hearing and also the five vestibular end organs for head movement sensing: the three semicircular canals (SCC) for head rotation and the two otolith organs, the utricule and saccule, for head translation. The three SCCs -- horizontal, anterior and posterior -- are named after their shape and position. Horizontal and superior canals are also called lateral and superior canals, respectively. Each SCC has an enlargement, the ampulla, containing the cupula, crista, sensory hair cells and innervating nerve fibers (Fig.\,\ref{fig:anatomy}B). 

There are type I and type II hair cells (Fernandez et al., 1988a). Both types are innervated by afferent neurons, cells that transmit nerve signals to the central nervous system and efferent neurons, cells that receive nerve signals from the central nervous system. The afferent neurons' cell bodies are inside the vestibular or Scarpa's ganglion (named after Italian anatomist Antonio Scarpa). From there the vestibular nerve transmits information to and from the vestibular nuclei in the brainstem. Detailed accounts of these parts can be found elsewhere, e.\,g. The Vestibular System, Angelaki and Cullen (2008), Eatock (2011) or Cullen (2011).
 
\subparagraph{Head movement mechanotransduction}
If the head is stationary, afferents in the SCC fire action potentials at a resting discharge rate, which in humans is typically 90 spikes per second (spk/s) (Goldberg and Fernandez, 1971a). This allows encoding of bi-directional rotation by increasing or decreasing the firing rate of afferents. For instance, a right head turn results in an increase (excitation) of afferent firing rate in the horizontal SCC of the right ear, whereas it leads to a decrease (inhibition) in the horizontal SCC of the left ear.
\hyphenation{laby-rinth}
\hyphenation{intra-laby-rinthine}

Specifically, the rotation is faithfully transmitted to the membranous labyrinth as it is fixed to the skull. However, the endolymph inside a SCC cannot follow the movement instantly due to its inertia and causes a deflection of the cupula. This induces a chain of biochemical events. Tiny deflections of hair bundles (100\,nm) inside the cupula modulate ion currents and influences vestibular afferents, ultimately leading to excitation of the vestibular nerve in one ear (Fig.\,\ref{fig:anatomy}C) and inhibition of the vestibular nerve in the opposite ear. Movement transduction has a bandpass characteristic and is well adapted to encode head velocities over the natural frequency range (0.5-20\,Hz, with peak around 2-5\,Hz).

Excitation and inhibition have been described as the 'push-pull' mechanism of the vestibular system and provides some redundancy (which allows unilateral vestibular loss patients to compensate). Three pairs are formed: the two horizontal canals, left anterior and right posterior (LARP), and right anterior and left posterior (RALP). Each pair measures rotation along a single axis. Due to their orthogonality, they can provide the brain with a three-dimensional reconstruction of rotational motion (Fig.\,\ref{fig:anatomy}D).  

Similarly, afferents in the otolith organs encode translational head motion. In contrast to the SCC, however, the utricle and saccule do not have distinct axes, but cover different directions. They additionally detect head tilt with respect to the earth's gravitational field.

\subparagraph{Afferent properties}
Goldberg and Fernandez (1971a) classified two categories of afferents. At rest, \emph{regular} afferents have relatively stable interspike intervals between action potentials, while  \emph{irregular} afferents have erratic intervals. This property is usually quantified with the normalized coefficient of variation $cv^*$, defined as the ratio of standard deviation of ISI over a standard mean interval (in mammals 15\,ms). These two afferent classes differ in various  aspects. Here, we only highlight their response dynamics and electrical excitability.

Afferent response dynamics have been investigated in-vivo and with a torsion-pendulum model  that models the fluid dynamics of the SCCs to calculate cupula deflection (Goldberg et al., 2012). Irregular afferents have higher sensitivity to rotational forces than regular afferents, while the latter have lower velocity detection thresholds. Irregular afferents best code high frequency movements. The movement transduction has a bandpass characteristic and is well adapted to encode head velocities over the natural frequency range (0.5-20\,Hz, with peak around 2-5\,Hz). Figure\,\ref{fig:anatomy}E shows the gain, i.\,e. change in afferent firing rate depending on head movement frequency. The mean maximum firing rate is approx. 500\,spk/s, though some irregular afferents can briefly reach 1000\,spk/s (Lysakowski et al., 1995). 

With respect to electrical excitability, irregular afferents have lower thresholds, i.\,e. are more sensitive to electrical stimulation with galvanic currents than their regular counterparts (likely due to irregular afferents' thicker axons) (Goldberg et al., 1984).

\subparagraph{Vestibulo-ocular reflex and optokinetic system}
The vestibulo-ocular reflex (VOR) is a reflexive eye movement that together with the optokinetic system (OKN) helps maintain a clear and stable view of the environment. Subjects with an impaired VOR have difficulty reading or recognizing objects while in motion. This impedes, for instance, independent driving. The reflex rotates our eyes in the opposite direction to any head motion and has both translational as well as rotational components. A head turn to the right causes excitation and inhibition of afferent firing rates of the right and left horizontal canals, respectively. This engages three nuclei and the oculomotor muscles to generate countering left eye movement (Fig.\,\ref{fig:anatomy}F). Because of its short pathway, often referred to as three neuron arc, the VOR is one of the fastest reflexes in our bodies with latencies between 5 and \SI{10}{\milli\second}.

For small head rotations (less than 15\degree), the VOR creates eye velocities matching head velocity. The VOR gain, the ratio of eye velocity over head velocity, in these cases equals 1. In these cases, the VOR has a predominantly linear behavior and has been described with linear control system tools (e.g., Robinson, 1981). For larger head movements, the VOR generates a nystagmus, a mixed strategy of slow and quick eye movements, to keep eye position within the oculomotor range.

Rotations solely around the horizontal axis result in predominantly horizontal VOR. Rotations along the LARP or RALP axis will yield vertical and also torsional VOR.

The OKN uses solely visual input to stabilize gaze and is driven by retinal slip, the relative motion of the visual world across the retina. Normally, OKN complements VOR at low-frequency head movements and produces an optokinetic nystagmus, i.\,e. slow compensatory and quick resetting eye movements.

\subparagraph{Peripheral vestibular disorders}
Chronic bilateral vestibular  loss can have traumatic, ototoxic, infectious, autoimmune or congenital causes. However, approximately half of the cases are idiopathic (Guinand et al., 2012a). The onset of BVL can be sudden, for instance following trauma, or gradually such as with the genetic disorder DFNA9 that leads to loss of auditory and vestibular function (Manolis et al., 1996).

In some clinical cases, antibiotics such as gentamicin can be used to treat vertigo attacks in Ménière's disease. But gentamicin is highly ototoxic. It leads to a loss of type I hair cells, renders type II hair cells insensitive to motion and also reduces resting discharge rates of afferents by 23\% in chinchillas (Hirvonen et al., 2005).
 
\section{Vestibular Implant Development}\label{sec:background:VI}
In comparison to deaf patients, BVL patients have currently no adequate treatment option. A vestibular implant could restore vestibular function in these patients. Both the improvement in their quality of lives as well as the economic benefit for society (Sun et al., 2014) justify the development of a VI.

We focus on VIs that specifically aim to restore SCC function. As we have learned in the previous section, otolithic organs encode various directions, in contrast to the distinct axes of SCCs. Therefore, replacement of otolithic function is deemed considerably more challenging since electrical stimulation would be insufficiently specific and would create sensations with various directions due to current spread in the target organ. In fact, one study suggested that providing solely rotational (SCC) information through a VI could also help with deficient otolith function (Sun et al., 2011).

\subparagraph{General concept}
A VI has to replace the process of mechanotransduction, that is, first, sensing head rotation and decomposing head velocities into the horizontal, LARP and RALP axes. Second, a controller has to compute the stimulation settings and a stimulator applies electrical stimulation to electrodes implanted in the labyrinth. Figure\,\ref{fig:viconcept}A visualizes the concept in a block diagram.
 
\subparagraph{Rotation sensor}
\hyphenation{micro-electro-mech-anical}
First VI protoypes employed single axis micro-electro-mech\-anical rotation sensors (Gong and Merfeld, 2000). These are devices similar to integrated circuits with dimensions between \SI{20}{\micro\metre} and \SI{1}{\milli\metre} and contain a processing and a sensor unit. The rotation sensors are typically based on the tuning forks or vibrating wheel, i.\,e. structures, that are actively oscillating  (forks or wheel), are displaced in their default plane of motion due to external forces. This displacement is then used to gauge rotation.

Today's MEMS gyroscopes are smaller, yet sense rotation in three axes, have improved sensitivity and larger measurement range compared to the sensor Gong and Merfeld used in 2000. The main remaining challenge is power consumption. Even at rest, the sensing structures have to maintain oscillation, therefore drawing ca. \SI{5}{\milli\ampere}. In comparison, a triple-axis accelerometer to measure linear acceleration uses ca. \SI{100}{\micro\ampere}, a 50-fold difference. However, a network of accelerometers would not accurately measure rotation (Park and Hong 2011). Ongoing research is directed at developing SCC-inspired rotation sensors with microfluidics (Andreou et al., 2014) that would consume less energy.
 
The rotation sensor has been typically placed externally in a head cap in animal models. For human patients, external placement inside the ear canal or rigidly fixed on the skull has been proposed as well as implantation (Garnham et al., 2012). Compared to the microphone in cochlear implants, the rotation sensor for a VI has to be secured in place to have fixed measurement axes. A fast and accurate procedure can align sensor measurements with SCC axes (DiGiovanna et al., 2012).

\subparagraph{Electrode design and surgical placement}
Electrodes were originally fashioned from twisted platinum wire with approx. \SI{75}{\micro\metre} diameter and \SI{200}{\micro\metre} length for guinea pigs (Gong and Merfeld, 2000). They distinguished between two configurations, \emph{monopolar} stimulation with one site in the labyrinth and a return electrode in the animal's neck musculature. In contrast, \emph{bipolar} stimulation would use an electrode inside the labyrinth as return electrode. 

Learning from experience as well as three-dimensional modeling of the inner ear, customized electrode arrays were manufactured for better placement. The different electrode sites with controlled spacing allow for redundancy, and also acquisition of VECAPs (Chiang et al., 2011; Poppendieck et al., 2014).  

Validation of these custom electrodes in humans would require substantial effort to obtain regulatory approval. Cochlear implants, on the other hand, offer established and approved electrode design and have up to 22 electrode sites. Thus, research groups have been adopting modified CIs that provide extra-cochlear electrode sites for vestibular stimulation (Perez-Fornos et al., 2014; Golub et al., 2013; Valentin et al., 2013). 

Different approaches have been established for the surgical placement of the electrodes (van de Berg et al., 2012). In the intralabyrinthine approach, electrodes are inserted close to the cristae of the ampullae. This avoids an injury of the middle ear and reduces the probability of activating the facial nerve during stimulation. However, electrical stimulation of the ampullae may result in an insufficient response as the corresponding nerve dendrites (fibers) may have died back towards scarpa's ganglion. In the extralabyrinthine approach, Guyot and co-workers successfully placed an electrode not close to the ampullae, but in a depression drilled near the posterior ampullary nerve branch in one patient.  

Vestibular implantation carries a considerable risk of hearing damage due the structure's proximity. Mild hearing loss occurred in rhesus monkeys (Dai et al., 2010) and some patients implanted with a VI suffered significant hearing impairment at the University of Washington (A. Perez-Fornos and N. Guinand, \textit{pers. comm.}). In contrast, CLONS collaborators in Geneva and in Maastricht have been conservative and only selected BVL patients also suffering from deafness.

Electrode insertion in animal models often renders the subjects vestibular deficient. This is also referred to as canal plugging and restricts endolymph movement that deactivates SCC function. In another approach, ototoxic antibiotic gentamicin has been injected to eliminate vestibular sensation (Della Santina et al., 2007).

To facilitate notation of electrode placement, electrodes placed close to or into the lateral ampullary nerve are called \emph{LAN}, for the superior and posterior ampullary nerves we will use \emph{SAN} and \emph{PAN}, respectively.
\begin{figure}[btp]
\centering
\includegraphics[width=.95\textwidth]{chapters/background/figures/Fig_VIconcept_v2.eps} 
\caption[Concept of vestibular implant]{Concept of VI, stimulation paradigms and VECAPs. (\textbf{A}) The concept of current vestibular implants. Head motion is sensed by a sensor such as gyroscope and/or accelerometers. A controller dissects measurements into the three rotational axes horizontal, LARP and RALP and sends corresponding stimulation pulses to the stimulator that applies electrical stimulation to the peripheral vestibular nerve. The stimulation activates the VOR pathway and eye movement response is used as benchmark for stimulation efficacy. (\textbf{B}) Most VIs use pulse rate modulation modeled with an equation similar to Eq.\,\ref{eq:prmgong}. Two shapes are shown here, shape 1 would be more sensitive to velocities between -200 and 200\,\degree /s than shape 2. (\textbf{C}) A VI uses biphasic, charge-balanced pulses with the parameters shown. (\textbf{D}) VECAPs from literature had characteristic negative (N) and positive waves (P). From Nie et al. (2011). (\textbf{E}) From Dai et al. (2011), an example of normal vestibular function in a rhesus monkey. Subsequently the researchers eliminated vestibular function with gentamicin and were able to restore it partially with the VI.
}
\label{fig:viconcept}
\end{figure}

\subparagraph{Stimulation paradigm}
Stimulation paradigms try to mimic the afferents' natural resting discharge rate and spike rate modulation. Specifically, electric pulses are applied with a baseline pulse rate and pulse rate modulation (PRM). To replicate the excitation and inhibition ('push-pull') mechanism of both ears, VIs should be ideally implanted bilaterally in BVL subjects. Bilateral implantation and stimulation has been reported for guinea pigs (Gong et al., 2008). However, due to the cost and risk of the surgery, instrumentation is restricted to one ear at this stage of VI development. The rationale is to restore vestibular function to a level comparable to unilateral vestibular loss patients that can compensate well in everyday life for their deficiency.

To create balanced space for excitation and inhibition, the baseline pulse rate has been elevated to 200 or 250 pulses per second (pps). This would give equivalent range to increase the pulse rate to 500\,pps (mean maximal natural firing rate 500\,spk/s) or to decrease it to 0\,pps. In case afferents had a natural resting discharge rate (e.\,g., 90\,spk/s), that rate would be the lower bound. Note the difference in units: We use spk/s for afferents' natural and spontaneous firing rate, while pps is used for induced firing rate by electrical stimulation. Merfeld et al. (2007) used Eq.\,\eqref{eq:prmgong} for PRM, see also Fig.\,\ref{fig:viconcept}B.
\begin{equation}\label{eq:prmgong}
pr = pr_{base} + pr_{range}\left[\tanh\left(\frac{\omega}{s_{shape}}\right)\right]
\end{equation}
where $pr$ is applied pulse rate, $pr_{base}$ and $pr_{range}$ the baseline pulse rate and range, respectively, $\omega$ the angular rate and $s_{shape}$ is a shape factor and determines for instance the region of linearity.

Electric pulses are applied as biphasic, charge-balanced and symmetric pulses. This pulse shape avoids electrode corrosion and evolution of gas or other toxic substances at the metal-saline interface due to faradaic and oxidation-reduction reactions (Rose and Robblee, 1990; Merrill et al., 2005). The leading phase is usually the cathodic phase (negative current amplitude) as this has been shown to be more effective in generating action potentials. Figure\,\ref{fig:viconcept}C shows the biphasic pulse and its parameters. 

Gong and Merfeld (2000) were guided by the original studies of Suzuki and Cohen in the 1960s and determined through trial and error a phase width and phase gap of \SI{200}{\micro\second}. This has been typically used in animal models as well as patients. A comprehensive study by Davidovics et al. (2011) found that both PRM as well as pulse amplitude modulation (PAM) effectively elicited eye movement responses. They also varied phase width systematically. The group stated in that PRM with short pulse widths should be "the foundation for further optimization". This was later refined to endorse co-modulation of both pulse amplitude and pulse rate (Davidovics et al., 2012, 2013). They found that responses depended on baseline pulse rate with low value favoring PRM, while superphysiological baseline pulse rates favored PAM. Similarly, Guyot et al. used a high baseline pulse rate of 400\,pps in instrumented patients and observed larger eye movement in response to PAM than PRM.

\subparagraph{Assessment of VI performance}
The VOR response is regarded as the most objectively measured output of the vestibular system. In animal models, search coils are implanted into the eye(s) (Gong and Merfeld, 2000). In a controlled electromagnetic field eye movement, and thus coil movement, can be recorded to measure eye movement. Videooculography is another means and has been employed in animal models as well as human patients. A camera mounted on special glasses or placed in front of the subject and corresponding software track the pupil's movement. Coil measurements currently provide higher sampling rates (e.\,g., $\SI{1}{\kilo\hertz}$) and better resolution. However, placement of coils can lead to edema and trauma that may distort eye movements.

Since the vestibular system also contributes to perception and posture, VI performance should be evaluated not only through the VOR response. Squirrel and rhesus monkeys were trained to execute a subjective visual vertical task and a balance perturbation task (Thompson et al., 2012). In the first task, subjects are tilted in a dark environment and need to align a visual bar to what they perceive as earth-vertical. In the second task, subjects stood on a balance platform in their natural quadropedal stance with different support surfaces (e.\,g., rubber foam) or in a wide and narrow stance. However, these tasks required substantial training and testing time with animal subjects. Perceptual tasks might be therefore more conveniently performed with human patients.

VECAPs could be another objective metric of vestibular function. They are a measure of gross action potential activation generated in response to electrical stimulation by a population of nerve fibers. Historically, ECAPs were measured in the auditory nerve of CI users (Brown and Abbas 1990) to investigate their clinical utility (Abbas et al., 1999). Possible applications include threshold prediction and automatic CI fitting (i.\,e., optimizing stimulation parameters). Today it is used to help setting stimulation parameters in infants and children (McKay et al., 2013).

VECAPs have been first reported 2011 as tool for electrode placement during surgery (Nie et al., 2011) and some correlation with VOR responses has been demonstrated (Dai et al., 2012). Figure\,\ref{fig:viconcept}D shows an example with the characterisitc negative peak N and positive peak P (literature also uses the term wave for peak, N and P waves). Recording VECAPs requires special techniques to reduce or even eliminate stimulation artifact which is generated due to proximity of stimulation and recording electrodes.

VECAPs occur within the first millisecond after onset of electrical stimulation. This is significantly faster than a VOR response and would be apt for a closed-loop VI. In Part\,\ref{part:vecap} we demonstrate acquisition of VECAPs and correlation with VOR responses in guinea pigs.  

\subparagraph{Restoration of vestibular function}
We have discussed vestibular neurophysiology and the components of a VI. Figure\,\ref{fig:viconcept}E shows the experimental chain for the horizontal canal in a rhesus monkey (Dai et al., 2011). First, the subject experienced whole body rotation around its lateral axis. VOR responses in this normal condition were predominantly horizontal. Second, sequential gentamicin treatment of the ears eliminated vestibular function. Third, with an activated VI, eye movement could be partially restored. However, there was asymmetry, i.\,e. the response was stronger in one head direction than the other, mimicking a unilateral vestibular impairment. Additionally, VOR responses along the undesired axes LARP and RALP were observed.

This illustrates remaining challenges in VI development: subpar VOR responses, misalignment (erroneous eye movement along non-desired axes) and asymmetry. Next, Part\,\ref{part:pamprm} investigates which stimulation paradigm evokes VOR responses more effectively during acute stimulation in patients. Part\,\ref{part:vecap} characterizes VECAPs in relation with VOR in guinea pigs that may lead to a closed-loop VI and to overall better implant performance. Part\,\ref{part:conclusions} summarizes findings and highlights remaining issues that need to be addressed in future VI development.