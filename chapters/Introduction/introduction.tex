\chapter{Introduction}\label{sec:introduction}
\epigraph{The senses do not err — not because they always judge rightly, but because they do not judge at all.}{\textit{Critique of Pure Reason}, Immanuel Kant}

\noindent Our senses accompany our everyday life: We hear an alarm ringing, we slowly open our eyes and touch our smartphone to turn the alarm off. The scent of fresh coffee helps us to rise out of bed, have breakfast and enjoy the wonderful taste of a pain au chocolat. 

This scene exemplifies the five senses that Aristotle (384 BC - 322 BC) is believed to have originally classified. The notion of a sixth sense consisting of the vestibular and other proprioceptive systems emerged only after the mid-19th century when scientists realized that vestibular organs in the inner ear were not related to hearing, but involved in equilibrium functions. Today, we are  more aware of the vestibular system's instrumental role in our everyday life.

A disruption of vestibular inputs can deteriorate motor coordination, postural control and spatial orientation. While individuals with unilateral vestibular deficiency or mild and moderate bilateral vestibular deficiency can at least partially compensate, patients with bilateral vestibular loss (BVL) have currently no viable treatment option. They can experience chronic dizziness, vertigo, imbalance or oscillopsia (blurred vision), thus significantly reducing their quality of life (TODO cite Sun et al 2014, Guinand et al 2012). 

\section*{Towards Vestibular Implants}
BVL patients may benefit from a vestibular implant (VI). Conceptually these are similar to cochlear implants that restore auditory function and have been the most successful neuroprosthesis to date with more than 300'000 people implanted worldwide (cite NIH report 2013). In a VI, electrodes or electrode arrays placed in the peripheral vestibular nerve branches, employing electrical stimulation to convey information about head movement to restore vestibular functionality (todo figure).

First experiments of electrical vestibular stimulation in 1960s. Suzuki, Cohen et al. activated vestibular structures with an implanted wire electrodes in cats (todo cite). They used monophasic constant voltage pulses, damage tissue (Shepherd 86 and '99) In 2000 first self-contained VI in guinea pig, one canal (Gong and Merfeld 2000, 2002). Extended to three canals (Della Santina 2007). A third group (Rubinstein). feasibility in humans Guyot et al 2009; Assessment, evaluation with Responses mediated through vestibular ocular reflex (VOR)

In 2009, CLONS started, ambitious, challenging, endeavor. Main objectives: tests in patients, animal models, assessment devices, modeling, most ambitious goal closed-loop. Number of partners involved. Regarding closed loop, concept briefly, vestibular signal or feedback signal to have cascaded closed-loop, identification and characterization of feedback signal. VECAP, vestibular electrically evoked compound action potential; investigation of correlation between VECAP and physiological signals in GPs.

In meantime human patients instrumented with modified cochlear implants, for instance in Geneva or Maastricht. Since patients both deaf and BVL, these were hybrid cochlear-vestibular implants with nine for cochlear and three independent electrode sites designated for vestibular stimulation. With approval to use modified cochlear implants, question arise which stimulation paradigm more effective: pulse amplitude (PAM) or pulse rate modulation (PRM). Goldberg and Fernandez' seminal studies found that primary afferents encode information with spike rate modulation (cite Goldberg and Fernandez). This predominantly emulated with PRM in implant prototypes in animal models. However, cochlear implants (orginally also employing PRM) commonly use PAM. 

\section*{Organization of this Thesis}
This short intro motivated research, described author's contribution to the field.

For the keen reader, Chap.\,\ref{sec:background} provides an overview of VI research since it has taken off in the 2000s.  

Part\,\ref{part:vecap} details how to reduce artifact and measure vecap (Chap.\,\ref{sec:artifact}, and presents correlation to VOR responses (Chap.\,\ref{sec:vecapvor}.

Part\,\ref{part:pamprm} presents the acute stimulation trials to investigate PAM and PRM in human patients. To this end, a customized real-time platform was programmed (Chap.\,\ref{sec:crio}), experimental results are discussed in Chap.\,\ref{sec:pamprm}.
TODO a figure to unify the two chapters here or later in part iii, better part iii

Finally, Part\,\ref{part:outlook} provides a summary as well as an outlook in the research field of vestibular implants.