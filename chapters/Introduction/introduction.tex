\chapter{Introduction}\label{chap:introduction}
%\chapter[toc version]{doc version}
%\chaptermark{version for header}

\epigraph{The senses do not err — not because they always judge rightly, but because they do not judge at all.}{\textit{Critique of Pure Reason}, Immanuel Kant, 1781}

\noindent Vision, touch, taste, hearing and smell are  the five senses that Aristotle (384 BC - 322 BC) is believed to have originally classified. The notion of a sixth sense consisting of the vestibular and other proprioceptive systems emerged only after the mid-19th century, when scientists realized that vestibular organs in the inner ear were not related to hearing, but involved in equilibrium functions. Today, we are  more aware of the vestibular system's instrumental role in our everyday life.

Each of our inner ears is equipped with five organs to sense head movement. Three semicircular canals sense head rotation and two otolith organs sense head translation.  A disruption of their inputs can deteriorate motor coordination, postural control and spatial orientation. While individuals with unilateral vestibular deficiency or mild and moderate bilateral vestibular deficiency can at least partially compensate, patients with bilateral vestibular loss (BVL) have currently no viable treatment option. They can experience chronic dizziness, vertigo, imbalance or oscillopsia (blurred vision), thus significantly reducing their quality of life (Sun et al., 2014; Guinand et al., 2012a). 

Guinand et al. (2012a) estimated the prevalence of BVL at 81 in 100'000 people, or 500'000 patients in Europe and the USA, three million worldwide. Though this number is considerably smaller than, for instance, for deaf people (approx. 70 million worldwide), BVL results in significant economic and social burdens, therefore warranting the development of a vestibular implant (VI) to restore vestibular function.

\section*{Towards Vestibular Implants}
VIs are conceptually similar to cochlear implants (CI) which restore auditory function. CIs have been the most successful neuroprosthesis to date with more than 300'000 people implanted worldwide (NIH report 2013). In a VI, electrodes or electrode arrays are placed in the peripheral vestibular nerve branches to apply electrical stimulation with the goal to convey information about head movement to the vestibular system.

Electrical stimulation of vestibular organs was first demonstrated in the 1960s. Suzuki, Cohen and colleagues activated vestibular structures with implanted wire electrodes in cats as well as monkeys (Cohen and Suzuki, 1963; Suzuki et al., 1969). The stimulation induced eye movement -- evidence of Vestibular-Ocular Reflex (VOR) activation. A healthy VOR contributes significantly to gaze stabilization and has been commonly regarded as the best objective measure of vestibular function.

In 2000, Gong and Merfeld succeeded in building the first self-contained, one-dimensional VI in a guinea pig (Gong and Merfeld, 2000). Their approach was then extended to cover all three rotational axes in chinchillas (Della Santina et al., 2007). The viability of a VI in human patients was demonstrated by Guyot and colleagues (Guyot et al., 2011ab). 

In 2009, the Automatic Control Laboratory at ETH Zurich and several  partners\footnote{Partners were: Scuola Superiore Sant'Anna, Fraunhofer Institute for Biomedical Engineering, University College of London, Centre National de la Recherche Scientifique, Hôpitaux Universitaires de Genève, Massachusetts Eye and Ear Infirmary, ETH Zurich, and from 2010 also Med-El. With the appointment of Prof. Micera to EPF Lausanne in 2011, work was transferred from Zurich to Lausanne.} started a four year project to pursue an innovative closed-loop neural prosthesis for vestibular disorders (CLONS). One ambition was to develop stimulation strategies to boost implant performance and aid neuroplasticity. The latter term generally describes the brain's (impressive) ability to adapt to changes in e.\,g., behavior, neural processes or environment. In several studies, animal subjects acclimated to continuous vestibular stimulation and showed improved responses with time that were attributed to plasticity (e.\,g., Merfeld et al., 2007, Dai et al., 2011). A closed-loop VI would hold promise to potentially accelerate plasticity and to enhance responses even further.

With several project partners, we began laying the groundwork for a closed-loop VI. Such a device would require a more immediate feedback signal than the naturally occurring visual feedback. The VOR typically has a latency of 5-10\,ms (Aw et al., 1996). To explore potential feedback signals, collaborators designed and implanted a double-sided electrode arrays in guinea pigs (Poppendieck et al., 2014). A total of eight electrode sites facilitated both stimulation and recording of the peripheral vestibular nerve. With those arrays  we identified vestibular electrically evoked compound action potentials (VECAPs, pronounced vee-e-kaps) as potential feedback signal and investigated it in relation to eye movement responses in guinea pigs. 

In a concurrent line of research, CLONS collaborators instrumented BVL patients, that also suffered from severe hearing loss, with hybrid cochlear-vestibular implants. These were modified CIs with nine electrode sites for cochlear and three electrode sites for vestibular stimulation. This approach provided established and approved CI technology as a possible fast-track to a commercial VI. However, it brought to focus the question of which stimulation paradigm was more effective: pulse amplitude (PAM) or pulse rate modulation (PRM). Goldberg and Fernandez' seminal studies found that primary afferents in the semicircular canals naturally encode information with spike rate modulation (Goldberg and Fernandez, 1971ab). This has been predominantly emulated with PRM in animal models. In contrast, stimulation strategies in CIs commonly employ PAM (Wilson et al., 2006). To compare both paradigms we tested them in four instrumented BVL patients during acute trials and strongly recommend PAM for the first time activation of a VI.

\section*{Layout of this Thesis}
This thesis focuses exclusively on VIs that aim to restore vestibular function through electrical stimulation, specifically the replacement of semicircular canal function to sense head rotation. Other approaches to restore vestibular sensation such as sensory substitution (Wall et al., 2009), infra-red stimulation (Harris et al., 2009), transcutaneous galvanic stimulation (Cohen et al., 2012), cell- or gene-based research to regrow or repair sensory cells (e.\,g., Koehler et al., 2013) are not discussed herein because of their limited or (still) uncertain prospects compared to electrical stimulation.  

The thesis details the author's contributions towards two aims: \textbf{Aim 1} identifying the more efficient electrical stimulation paradigm for initial VI activation in human patients, and \textbf{Aim 2} characterizing VECAPs in an animal model for a closed-loop VI that could potentially improve implant performance. 

Following this introduction, Chapter\,\ref{chap:background} provides an overview of vestibular neurophysiology and VI research that has seen steady and promising progress over the last 15 years. 

Part\,\ref{part:pamprm} presents the acute stimulation trials to investigate PAM and PRM stimulation paradigms in BVL patients. To this end, a customized real-time platform was programmed and deployed, described in Chapter\,\ref{chap:crio}. Experimental results and an accompanying neural network model are presented and discussed in Chapter\,\ref{chap:pamprm}. 

In Part\,\ref{part:vecap}, Chapter\,\ref{chap:artifact} reports how to reduce stimulation artifact to measure VECAP. Chapter\,\ref{chap:vecapvor} presents VECAP characterization for single electrical pulses, pulse trains and continuous electrical stimulation and how VECAP correlated with VOR responses in guinea pigs.

Part\,\ref{part:conclusions} discusses the universal themes emerging from animal and human testing. Finally, an outlook suggests the critical research questions that have to be addressed for VIs to become a success.