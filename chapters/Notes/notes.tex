\chapter{Notes}\label{sec:notes}
\paragraph*{Pulse Characteristic}
biphasic, charge-balanced (Bonham and Litvak, 2008) to avoid electrode corrosion and evolution of gas or other toxic substances that occur at metal-saline interface due to Faradaic, oxidation-reduction reactions (Rose and Robblee, 1990; Merrill et al., 2005)

first phase typically cathodic, excitatory for neural tissue near extracellular electrode, anodic for charge balance

\paragraph*{current steering?}

\paragraph*{FEM Modeling}
Hayden et al. 2011 for chinchillas, Hedjoudje et al. 2012 for rhesus monkeys, Prisca et al. for humans

fitting experimental data, electrode design optimization

monopolar stimulation, strong activation with target nerve branch but leading to significant current spread; bipolar stimlulation more local, transverse to axis of ampulla and perpendicular to distal trajectory of target nerve ??

\paragraph*{Charge delivery}
it is well known that charge delivery is the most crucial parameter to control for electrical stimulation of neurons. Charge delivery is simply the time integral of current. A voltage source cannot easily maintain con- stant current or constant charge delivery since changes in tissue impedance over time will lead to changes in the current and, hence, the charge delivered to the nerve. Furthermore, controlling current, and hence charge deliv- ery, also makes irreversible electrode dissolution less likely.

\paragraph*{Alternative stimulation, infrared}
Infrared (IR) laser stimulation has recently garnered interest as a potential means to stimulate the cochlear nerve with spatial selectivity much greater than that achieved using electrical stimulation via electrodes implanted within the scala tympani (Richter et al., 2011). Rajguru et al. (2011) extended the approach to the vestibular labyrinth in toadfish, and Ahn et al. (2012) recently extended the same approach to the mammalian labyrinth. Contrary to the results obtained in mammalian cochlea and in other peripheral nerves (Wells et al., 2007), IR laser stimulation applied vestibu- lar afferent neuron axons and somata yields little to no response in either toadfish or chinchillas in experiments published so far; however, IR directed at the crista (and presumably acting at vestibular hair cells) elicits a vari- ety of changes in afferent firing rates, including excitatory, inhibitory, and mixed responses. Responses observed using single-unit recording techniques clearly demonstrate phase-locking to individual laser pulses, and hence this effect cannot be owing solely to the bulk movement of labyrinthine fluids that occurs during clini- cal caloric nystagmography testing.
Whether IR laser stimulation will prove useful as a therapeutic means of prosthetic stimulation is as yet unclear. VOR responses have not yet been demonstrated in response to IR stimuli, and despite the large changes of firing rate observed in individual vestibular afferents, the fact that IR can evoke an uncontrollable combination of excitatory, inhibitory, and mixed responses within dif- ferent afferents within a given ampullary nerve will at least complicate the use of IR to encode head rotation. Nonetheless, IR laser stimulation may as yet have a therapeutic role, particularly if its high spatial resolu- tion can be leveraged to achieve selective stimulation of many small patches in the utricle and saccule. Optical stimulation (whether IR or other spectral ranges) may also serve as an adjunct in combination with electrical stimulation or through incorporation of photosensitive channels via optogenetic techniques (Wang et al., 2012).
Seeking another alternative to electrical stimulation, Merfeld et al. described in concept a mechanical vestibu- lar stimulator intended to amplify cupular movement via a piston or balloon implanted within a SCC. Although analogous mechanisms have long been used to great effect in bench research preparations to provide a ‘‘natural’’ stimulus to a SCC without having to rotate an animal (Rabbitt et al., 1995), the potential for clinical application of an implanted mechanical vestibular stimu- lator is unclear. Considering that most individuals with less-than-profound loss of vestibular sensation compen- sate adequately through reliance on other senses, the need for an implantable mechanical vestibular stimula- tor is less clear than the need for an electrical device
that can address profound sensory loss. Such devices may therefore face the same sort of economic and medi- cal decision-making constraints that have left little room for middle ear implantable hearing aids between conven- tional hearing aids and cochlear implants.
Various sensory substitution techniques have been advocated for conveying information about head motion to the central nervous system, including auditory stimuli delivered via headphones (Dozza et al., 2005; Hegeman et al., 2005), cutaneous sensation delivered by trunk vibrators (Wall, 2010), and electrotactile stimulation of the tongue (Tyler et al., 2003). Although each of these approaches has the potential to provide limited informa- tion about postural sway and head movement, none presents information to the central nervous system with sufficient resolution, specificity or speed to recreate a normal 3D VOR.