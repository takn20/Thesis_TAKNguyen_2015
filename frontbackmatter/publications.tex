\pdfbookmark[1]{Publications}{Publications}
\chapter*{Publications}

The following publications are included in parts or in an extended version in
this thesis:

Part\,\ref{part:pamprm}
\begin{itemize}
\item Nguyen TAK, Ranieri M, DiGiovanna J, Peter O, Genovese V, Perez Fornos A, and Micera S. A Real-Time Research Platform to Study Vestibular Implants With Gyroscopic Inputs in Vestibular Deficient Subjects. Biomedical Circuits and Systems, IEEE Transactions on 8: 474-484, 2014.

\item Nguyen TAK, DiGiovanna J, Cavuscens S, Ranieri M, van de Berg R, Guinand N, Carpaneto J, Kingma H, Guyot JP, Micera S, Pérez-Fornos A. Pulse Amplitude Modulation Is More Effective Than Pulse Rate Modulation during Acute Electrical Stimulation with a Vestibular Implant in Human Patients. Brain (in preparation), 2015.
\end{itemize}

Part\,\ref{part:vecap}
\begin{itemize}
\item Nguyen TAK, DiGiovanna J, Merfeld D, and Micera S. Comparing Artifact Reduction Methods for Recording Vestibular Evoked Potentials in a Vestibular Neuroprosthesis. In: 34th Annual International Conference of the EMBS. San Diego, CA: 2012.
\item Nguyen TAK, Gong W, DiGiovanna J, Poppendieck W, and Micera S. Investigating Vestibular Evoked Potentials as Feedback Signal in a Vestibular Neuroprosthesis: Relation to Eye Movement Velocity. In: Converging Clinical and Engineering Research on Neurorehabilitation, edited by Pons JL, Torricelli D, and Pajaro MSpringer Berlin Heidelberg, 2013a, p. 1325-1329.
\item Nguyen TAK, Gong W, Poppendieck W, DiGiovanna J, and Micera S. Investigating ocular movements and Vestibular Evoked Potentials for a vestibular neuroprosthesis: Response to pulse trains and baseline stimulation. In: 2013 6th International IEEE/EMBS Conference on Neural Engineering (NER). San Diego, CA, USA: 2013b.
\item Nguyen TAK, DiGiovanna J, Gong W, Haburcakova C, Poppendieck W, Micera S, Merfeld D. Characterizing Vestibular Electrically Evoked Compound Action Potentials in Guinea Pigs. In: 38th Annual MidWinter Meeting of the Association for Research in Otolaryngology. Baltimore, MD, USA, 2015.
\end{itemize}
%    \item \fullcite{Pletscher2009}.
%    \item \fullcite{Pletscher2010}.
%    \item \fullcite{Pletscher2011}.
%    \item \fullcite{Pletscher2011a}.
%    \item \fullcite{Pletscher2012a}.
%    \item \fullcite{Pletscher2012b}.
%    \item \fullcite{Pletscher2012c}.


Furthermore, the following publications were part of my PhD research, are
however not covered in this thesis. The first two publications were early animal recordings with a setup different from the one used in the main section. The topics of the other two publications are outside of the scope of the material covered here:
\begin{itemize}
\item Kögler V, Nguyen TAK, DiGiovanna J, and Micera S. Recording Vestibular Evoked Potentials Induced by Electrical Stimulation of the Horizontal Semicircular Canal in Guinea Pig. In: IEEE EMBS Conference on Neural Engineering. Cancun: 2011.
\item Nguyen TAK, Kögler V, DiGiovanna J, and Micera S. Finding Physiological Responses in Vestibular Evoked Potentials. In: IEEE Engineering in Medicine and Biology Conference. Boston, MA, USA: 2011.
\item Poppendieck W, Sossalla A, Krob M-O, Welsch C, Nguyen TAK, Gong W, DiGiovanna J, Micera S, Merfeld D, and Hoffmann K-P. Development, manufacturing and application of double-sided flexible implantable microelectrodes. Biomedical Microdevices 1-14, 2014.
\item Van De Berg R, Guinand N, Nguyen TAK, Ranieri M, Cavuscens S, Guyot J-P, Stokroos R, Kingma H, and Perez Fornos A. The vestibular implant: Frequency-dependency of the electrically evoked Vestibulo-Ocular Reflex in humans. Frontiers in Systems Neuroscience 8: 2015.
%    \item \fullcite{Brand2008}.
%    \item \fullcite{Jaggi2012}.
%    \item \fullcite{LacosteJulien2013}.
\end{itemize}
