\manualmark
\markboth{\spacedlowsmallcaps{\bibname}}{\spacedlowsmallcaps{\bibname}} % work-around to have small caps also
\refstepcounter{dummy}
\addtocontents{toc}{\protect\vspace{\beforebibskip}} % to have the bib a bit from the rest in the toc
\addcontentsline{toc}{chapter}{\tocEntry{\bibname}}
%\printbibliography
\chapter*{\bibname}\label{app:bibliography}
{\footnotesize
\indent Abbas P, Brown C, Shallop J, Firszt J, Hughes M, Hong S, and Staller S. Summary of results using the nucleus CI24M implant to record the electrically evoked compound action potential. Ear and Hearing 20: 45-59, 1999.

Andreou C, Pahitas Y, and Georgiou J. Bio-Inspired Micro-Fluidic Angular-Rate Sensor for Vestibular Prostheses. Sensors 14: 13173-13185, 2014.

Angelaki DE, and Cullen KE. Vestibular system: The many facets of a multimodal sense. Annual Review of Neuroscience 31: 125-150, 2008.

Arnold DB, and Robinson DA. The oculomotor integrator: testing of a neural network model. Experimental Brain Research 113: 57-74, 1997.

Aw S, Haslwanter T, Halmagyi G, Curthoys I, Yavor R, and Todd M. Three-dimensional vector analysis of the human vestibuloocular reflex in response to high-acceleration head rotations .1. Responses in normal subjects. Journal of Neurophysiology 76: 4009-4020, 1996.

Bahmer A, Peter O, and Baumann U. Recording and analysis of electrically evoked compound action potentials (ECAPs) with MED-EL cochlear implants and different artifact reduction strategies in Matlab. Journal of Neuroscience Methods 191: 66-74, 2010.

Baloh RW, and Kerber KA editors. Clinical Neurophysiology of the Vestibular System. Oxford University Press, 2011.

Baumann U, and Nobbe A. The cochlear implant electrode–pitch function. Hearing Research 213: 34-42, 2006.

Brown CJ, Abbas PJ, and Gantz B. Electrically evoked whole-nerve action potentials: Data from human cochlear implant users. The Journal of the Acoustical Society of America 88: 1385-1391, 1990.

Carpaneto J, Genovese V, Ghionzoli R, Lewis R, Merfeld D, Sabatini A, and Micera S. Characterization and calibration of a novel 3 axis gyroscope for a vestibular neuroprosthesis (in preparation). Medical Engineering \& Physics.

Cohen B, and Suzuki J-I. Eye movements induced by ampullary nerve stimulation. American Journal of Physiology -- Legacy Content 204: 347-351, 1963.

Cohen B, Yakushin SB, and Holstein GR. What does galvanic vestibular stimulation actually activate: response. Frontiers in Neurology 3: 2012.

Constandinou TG, Georgiou J, and Toumazou C. A Partial-Current-Steering Biphasic Stimulation Driver for Vestibular Prostheses. Biomedical Circuits and Systems, IEEE Transactions on 2: 106-113, 2008.

Cremer PD, Halmagyi GM, Aw ST, Curthoys IS, McGarvie LA, Todd MJ, Black RA, and Hannigan IP. Semicircular canal plane head impulses detect absent function of individual semicircular canals. 1998, p. 699-716.

Cullen KE. The neural encoding of self-motion. Current Opinion in Neurobiology In Press, Corrected Proof: -, 2011.

Curthoys IS. Vestibular compensation and substitution. Current Opinion in Neurology 13: 27-30, 2000.

Dai C, Ahn JH, Fridman G, Rahman MA, and Della Santina CC. Electrically-evoked Compound Action Potentials and 3D Vestibulo-ocular Reflex Are Correlated to Each Other and to Electrode Position in Rhesus Monkeys Using a Multichannel Vestibular Prosthesis. In: 35th MidWinter Meeting Association of Research in Otolaryngology. San Diego: 2012.

Dai C, Fridman G, Chiang B, Davidovics N, Melvin T-A, Cullen K, and Della Santina C. Cross-axis adaptation improves 3D vestibulo-ocular reflex alignment during chronic stimulation via a head-mounted multichannel vestibular prosthesis. Experimental Brain Research 1-12, 2011.

Davidovics N, Fridman G, and Della Santina C. Co-modulation of stimulus rate and current from elevated baselines expands head motion encoding range of the vestibular prosthesis. Experimental Brain Research 218: 389-400, 2012.

Davidovics N, Rahman M, Dai C, Ahn J, Fridman G, and Della Santina C. Multichannel Vestibular Prosthesis Employing Modulation of Pulse Rate and Current with Alignment Precompensation Elicits Improved VOR Performance in Monkeys. JARO 14: 233-248, 2013.

Davidovics NS, Fridman GY, Chiang B, and Della Santina CC. Effects of Biphasic Current Pulse Frequency, Amplitude, Duration, and Interphase Gap on Eye Movement Responses to Prosthetic Electrical Stimulation of the Vestibular Nerve. IEEE Transactions on Neural Systems and Rehabilitation Engineering 19: 84 -94, 2011.

Della Santina CC, Migliaccio AA, and Patel AH. A multichannel semicircular canal neural prosthesis using electrical stimulation to restore 3-D vestibular sensation. IEEE Transactions on Biomedical Engineering 54: 1016-1030, 2007.

Eatock RA, and Songer JE. Vestibular Hair Cells and Afferents: Two Channels for Head Motion Signals. Annual Review of Neuroscience 34: 501-534, 2011.

Fetter M, and Dichgans J. Vestibular neuritis spares the inferior division of the vestibular nerve. 1996, p. 755-763.

Fridman GY, Davidovics NS, Dai C, Migliaccio AA, and Della Santina CC. Vestibulo-Ocular Reflex Responses to a Multichannel Vestibular Prosthesis Incorporating a 3D Coordinate Transformation for Correction of Misalignment. JARO-Journal of the Association for Research in Otolaryngology 11: 367-381, 2010.

Fridman GY, and Della Santina CC. Progress Toward Development of a Multichannel Vestibular Prosthesis for Treatment of Bilateral Vestibular Deficiency. The Anatomical Record: Advances in Integrative Anatomy and Evolutionary Biology 295: 2010-2029, 2012.

Goldberg J, and Fernandez C. Physiology of Peripheral Neurons Innervating Semicircular Canals of Squirrel Monkey. I. Resting Discharge and Response to Constant Angular Accelerations. Journal of Neurophysiology 34: 635-660, 1971a.

Goldberg J, and Fernandez C. Physiology of Peripheral Neurons Innervating Semicircular Canals of Squirrel Monkey. III. Variations among Units in Their Discharge Properties. Journal of Neurophysiology 34: 676-684, 1971b.

Goldberg JM, Wilson VJ, Cullen K, Angelaki D, Broussard DM, Büttner-Ennever JA, Fukushima K, and Minor L editors. The Vestibular System: A Sixth Sense. Oxford University Press, 2012.

Gong W, Haburcakova C, and Merfeld DM. Vestibulo-Ocular Responses Evoked Via Bilateral Electrical Stimulation of the Lateral Semicircular Canals. IEEE Transactions on Biomedical Engineering 55: 2608-2619, 2008.

Gong W, and Merfeld D. Prototype neural semicircular canal prosthesis using patterned electrical stimulation. Annals of Biomedical Engineering 28: 572-581, 2000.

Grossman GE, Leigh RJ, Abel LA, Lanska DJ, and Thurston SE. Frequency and velocity of rotational head perturbations during locomotion. Experimental Brain Research 70: 470-476, 1988.

Guinand N, Boselie F, Guyot JP, and Kingma H. Quality of life of patients with bilateral vestibulopathy. The Annals of otology, rhinology \& laryngology 121: 471-477, 2012a.

Guinand N, Pijnenburg M, Janssen M, and Kingma H. Visual Acuity While Walking and Oscillopsia Severity in Healthy Subjects and Patients With Unilateral and Bilateral Vestibular Function Loss. Arch Otolaryngol Head Neck Surg 138: 301-306, 2012b.

Guyot J-P, Sigrist A, Pelizzone M, Feigl GC, and Kos MI. Eye Movements in Response to Electrical Stimulation of the Lateral and Superior Ampullary Nerves. Annals of Otology Rhinology and Laryngology 120: 81-87, 2011a.

Guyot J-P, Sigrist A, Pelizzone M, and Kos MI. Adaptation to Steady-State Electrical Stimulation of the Vestibular System in Humans. Annals of Otology Rhinology and Laryngology 120: 143-149, 2011b.

Harris DM, Bierer SM, Wells JD, and Phillips JO. Optical nerve stimulation for a vestibular prosthesis. In: Proc SPIE2009, p. 7180-7121.

Hayden R, Sawyer S, Frey E, Mori S, Migliaccio A, and Della Santina C. Virtual labyrinth model of vestibular afferent excitation via implanted electrodes: validation and application to design of a multichannel vestibular prosthesis. Experimental Brain Research 1-18, 2011.

Hirvonen TP, Minor LB, Hullar TE, and Carey JP. Effects of Intratympanic Gentamicin on Vestibular Afferents and Hair Cells in the Chinchilla. Journal of Neurophysiology 93: 643-655, 2005.

Koehler KR, Mikosz AM, Molosh AI, Patel D, and Hashino E. Generation of inner ear sensory epithelia from pluripotent stem cells in 3D culture. Nature 500: 217-221, 2013.
Lewis R, Gong W, Ramsey M, Minor L, Boyle R, and Merfeld D. Vestibular adaptation studied with a prosthetic semicircular canal. Journal of Vestibular Research-Equilibrium \& Orientation 12: 87-94, 2002.

Lewis RF, Haburcakova C, Gong W, Makary C, and Merfeld DM. Vestibuloocular Reflex Adaptation Investigated With Chronic Motion-Modulated Electrical Stimulation of Semicircular Canal Afferents. Journal of Neurophysiology 103: 1066-1079, 2010.

Lysakowski A, Minor LB, Fernandez C, and Goldberg JM. Physiological identification of morphologically distinct afferent classes innervating the cristae ampullares of the squirrel monkey. 1995, p. 1270-1281.

Manolis EN, Yandavi N, Nadol JB, Eavey RD, McKenna M, Rosenbaum S, Khetarpal U, Halpin C, Merchant SN, Duyk GM, MacRae C, Seidman CE, and Seidman JG. A Gene for Non-Syndromic Autosomal Dominant Progressive Postlingual Sensorineural Hearing Loss Maps to Chromosome 14q12–13. Human Molecular Genetics 5: 1047-1050, 1996.

Marianelli P, Capogrosso M, Bassi Luciani L, Panarese A, and Micera S. A Computational Framework for Electrical Stimulation of Vestibular Nerve. Transactions on Neural Systems and Rehabilitation Engineering PP: PP, 2015 (accepted).

McKay C, Chandan K, Akhoun I, Siciliano C, and Kluk K. Can ECAP Measures Be Used for Totally Objective Programming of Cochlear Implants? JARO 14: 879-890, 2013.

Merfeld DM, Gong W, Morrissey J, Saginaw M, Haburcakova C, and Lewis RF. Acclimation to chronic constant-rate peripheral stimulation provided by a vestibular prosthesis. IEEE Transactions on Biomedical Engineering 53: 2362-2372, 2006.

Merfeld DM, Haburcakova C, Gong W, and Lewis RF. Chronic vestibulo-ocular reflexes evoked by a vestibular prosthesis. IEEE Transactions on Biomedical Engineering 54: 1005-1015, 2007.

Merfeld DM, and Lewis RF. Replacing semicircular canal function with a vestibular implant. Current Opinion in Otolaryngology \& Head and Neck Surgery 20: 386-392 310.1097/MOO.1090b1013e328357630f, 2012.

Minor L, and Goldberg J. Vestibular-nerve inputs to the vestibulo-ocular reflex: a functional- ablation study in the squirrel monkey. The Journal of Neuroscience 11: 1636-1648, 1991.

Nguyen K, Kögler V, DiGiovanna J, and Micera S. Finding Physiological Responses in Vestibular Evoked Potentials. In: IEEE Engineering in Medicine and Biology Conference. Boston, MA, USA: 2011.
Nguyen TA, Ranieri M, DiGiovanna J, Peter O, Genovese V, Perez Fornos A, and Micera S. A Real-Time Research Platform to Study Vestibular Implants With Gyroscopic Inputs in Vestibular Deficient Subjects. Biomedical Circuits and Systems, IEEE Transactions on 8: 474-484, 2014.

Nie K, Bierer SM, Ling L, Oxford T, Rubinstein JT, and Phillips JO. Characterization of the Electrically Evoked Compound Action Potential of the Vestibular Nerve. Otology \& Neurology 32: 88-97, 2011.

Nie K, Ling L, Bierer S, Kaneko C, Fuchs A, Oxford T, Rubinstein J, and Phillips J. An Experimental Vestibular Neural Prosthesis: Design and Preliminary Results with Rhesus Monkeys Stimulated with Modulated Pulses. Biomedical Engineering, IEEE Transactions on PP: 1-1, 2013.

Park S, and Hong SK. Angular Rate Estimation Using a Distributed Set of Accelerometers. Sensors 11: 10444-10457, 2011.

Phillips JO, Bierer SM, Ling L, Kaibao N, and Rubinstein JT. Real-time communication of head velocity and acceleration for an externally mounted vestibular prosthesis. In: Engineering in Medicine and Biology Society,EMBC, 2011 Annual International Conference of the IEEE2011, p. 3537-3541.

Poppendieck W, Sossalla A, Krob M-O, Welsch C, Nguyen TAK, Gong W, DiGiovanna J, Micera S, Merfeld D, and Hoffmann K-P. Development, manufacturing and application of double-sided flexible implantable microelectrodes. Biomedical Microdevices 1-14, 2014.

Pozzo T, Berthoz A, and Lefort L. Head stabilization during various locomotor tasks in humans. Experimental Brain Research 82: 97-106, 1990.

Pozzo T, Berthoz A, Lefort L, and Vitte E. Head stabilization during various locomotor tasks in humans. Experimental Brain Research 85: 208-217, 1991.

Robinson DA. The Use of Control Systems Analysis in the Neurophysiology of Eye Movements. Annual Review of Neuroscience 4: 463-503, 1981.

Rosengren SM, Welgampola MS, and Colebatch JG. Vestibular evoked myogenic potentials: Past, present and future. Clinical Neurophysiology 121: 636-651, 2010.

Rubinstein JT, Bierer S, Kaneko C, Ling L, Nie K, Oxford T, Newlands S, Santos F, Risi F, Abbas PJ, and Phillips JO. Implantation of the Semicircular Canals With Preservation of Hearing and Rotational Sensitivity: A Vestibular Neurostimulator Suitable for Clinical Research. Otology \& Neurotology 33: 789-796 710.1097/MAO.1090b1013e318254ec318224, 2012.

Saginaw MA, Wangsong G, Haburcakova C, and Merfeld DM. Attenuation of Eye Movements Evoked by a Vestibular Implant at the Frequency of the Baseline Pulse Rate. Biomedical Engineering, IEEE Transactions on 58: 2732-2739, 2011.

Sun D, Ward BK, Semenov Y, Carey J, and Della Santina CC. Bilateral Vestibular Deficiency: Quality of Life and Economic Implications. JAMA Otolaryngol Head Neck Surg 140: 527-534, 2014.

Sun DQ, Rahman MA, Fridman G, Chenkai D, Chiang B, and Della Santina CC. Chronic stimulation of the semicircular canals using a multichannel vestibular prosthesis: Effects on locomotion and angular vestibulo-ocular reflex in chinchillas. In: Engineering in Medicine and Biology Society, EMBC, 2011 Annual International Conference of the IEEE2011, p. 3519-3523.

Suzuki J, Goto K, Tokumasu K, and Cohen B. Implantation of Electrodes near Individual Vestibular Nerve Branches in Mammals. Annals of Otology Rhinology and Laryngology 78: 1969.

Thompson LA, Haburcakova C, Gong W, Lee DJ, Wall III C, Merfeld DM, and Lewis RF. Responses evoked by a vestibular implant providing chronic stimulation. Journal of Vestibular Research 22: 11-15, 2012.

Toreyin H, and Bhatti P. A Field-Programmable Analog Array Development Platform for Vestibular Prosthesis Signal Processing. Biomedical Circuits and Systems, IEEE Transactions on 7: 319-325, 2013.

Van de Berg R, Guinand N, Guyot J-P, Stokroos R, and Kingma H. The vestibular implant: Quo vadis? Frontiers in Neurology 2: 2011.

Wall I, Conrad, Wrisley DM, and Statler KD. Vibrotactile tilt feedback improves dynamic gait index: A fall risk indicator in older adults. Gait \& Posture 30: 16-21, 2009.

Wilson BS. Speech processing strategies. Cochlear implants: a practical guide 2: 21-69, 2006.
Zierhofer CM. Adaptive sigma-delta modulation with one-bit quantization. Circuits and Systems II: Analog and Digital Signal Processing, IEEE Transactions on 47: 408-415, 2000.

Zingler VC, Weintz E, Jahn K, Mike A, Huppert D, Rettinger N, Brandt T, and Strupp M. Follow-up of vestibular function in bilateral vestibulopathy. Journal of Neurology, Neurosurgery \& Psychiatry 79: 284-288, 2008.
}