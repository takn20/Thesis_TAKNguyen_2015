\pdfbookmark[1]{Abstract}{Abstract}
\begingroup
\let\clearpage\relax
\let\cleardoublepage\relax
\let\cleardoublepage\relax

\chapter*{Abstract}
Every day we use our five senses to interact with our environment. A sixth sense -- the sense of balance  -- helps us navigate and orient through this environment. The vestibular system is the main contributor to this sense. Its sensory organs are located in the inner ear. Bilateral loss of vestibular sensation (BVL) is severely debilitating and significantly reduces quality of life. Affected patients have currently no effective treatment option and a vestibular implant (VI) might be the most promising option.

The field of vestibular implant research is entering an exciting period. More than a decade of animal work is now being translated to human patients and studies with patients are surging forward. With several collaborators within the EU and US project CLONS, we made further inroads towards a commercial VI. 

In the first part of this thesis, we developed a real-time research platform to investigate pulse rate and pulse amplitude modulation. These two modulation paradigms were then tested in four BVL patients which were instrumented with modified cochlear implants for vestibular stimulation. Our results strongly support pulse amplitude modulation as stimulation paradigm for implant activation because it evoked significantly stronger eye movement responses. Furthermore, a model of the eye movement reflex pathway provided insights into how differently the two modulation paradigms engage vestibular afferents. Future experiments should investigate how the performance of these modulation paradigms changes with continuous stimulation as animal research suggests improvements in eye movement response over time.

In the second part of the thesis, we lay the groundwork to pursue the goal of a closed-loop VI that could boost implant performance. We adapted artifact reduction techniques to record vestibular electrically evoked compound action potentials with a custom electrode array. We then correlated these compound potentials to eye movement responses for different stimulation scenarios that would resemble the operating mode of a VI. Our results showed a (piecewise) linear pattern that could be used clinically for implant fitting (i.\,e. identification of stimulation thresholds). Our results also showed that utilization of these compound potentials as feedback signal in a closed-loop VI may be feasible and requires further work.  

 
%\vskip 5cm
%
\vfill

\pagebreak

\selectlanguage{ngerman}

\pdfbookmark[1]{Zusammenfassung}{Zusammenfassung}
\chapter*{Zusammenfassung}
Wir nutzen täglich unsere fünf Sinne, um mit unserer Umgebung zu interagieren. Unser sechster Sinn -- der Gleichgewichtssinn -- hilft, dass wir uns in dieser Umgebung navigieren und orientieren können. Das Vestibulärsystem trägt massgebend dazu bei. Ein beidseitiger Verlust des Vestibulärfunktion ist eine schwerwiegende Behinderung und mindert nachweislich die Lebensqualität. Betroffene Patienten hatten jedoch bisher keine Therapiemöglichkeit. Ein Vestibulärimplantat (VI) ist ihre grösste Chance.

Das Forschungsfeld der VI-Forschung steht vor einer spannenden Phase. Mehr als zehn Jahre Forschung an Tieren wird derzeit auf menschliche Patienten übertragen und Studien mit Patienten nehmen zu. Im Rahmen des EU-US Projekts CLONS, machten wir mit unseren Projektpartnern weitere Fortschritte bzgl. eines Vestibulärimplantats.

Im ersten Teil der Dissertation entwickelten wir eine Echtzeitplattform, um Pulsraten- und Pulsweitenmodulation für ein VI in Patienten zu untersuchen. Die Patienten hatten bereits ein modifiziertes Cochlearimplantat erhalten, das Elektroden zur elektrischen Stimulierung des Vestibulärnervs enthielt. Unsere Ergebnisse zeigten deutlich, dass Pulsamplitudenmodulation in akuten Tests stärkere Augenbewegungen auslöste als Pulsratenmodulation und daher beim erstmaligen Einschalten des Implantats ausgewählt werden sollte. Unser Modell deckte überdies auf, wie die beiden Modulationstechniken Vestibulärneuronen aktiviert. Zukünftige Experimente müssen klären, ob und um wieviel sich die Performanzen der beiden Modulationsarten bei chronischer Stimulierung ändern. Literatur legt nahe, dass sich Augenbewegungen verstärken würden.

Im zweiten Teil der Dissertation legten wir den Grundstein für die Entwicklung einer Regelschleife für ein VI, welche die Leistung des Implantats weiter steigern könnte. Techniken zur Artefaktreduzierung wurden modifiziert, um vestibuläre evozierte Potenziale mit einem doppelseitigen Elektrodenarray zu messen (VECAPs). Wir korrelierten diese VECAPs mit den durch elektrische Stimulierung ausgelösten Augenbewegungen. Die Ergebnisse zeigten für ein Szenario ein stückweises, lineares Muster, das zur Schwellwertbestimmung für das Implantat verwendet werden könnte. Ausserdem zeigen unsere Resultate, dass VECAP potenziell rückführende Regelgrösse sein könnte für ein VI mit Regelkreis. Jedoch ist weitere Forschung dafür nötig.


\selectlanguage{american}

\endgroup

\vfill